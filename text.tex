\documentclass[12pt, a4paper]{article}
\usepackage[T2A]{fontenc}
\usepackage[utf8]{inputenc}
\usepackage[english,russian]{babel}
\usepackage{color}
\usepackage{listings}
\let\stdsection\section
\renewcommand\section{\newpage\stdsection}
\definecolor{lightgray}{rgb}{.9,.9,.9}
\definecolor{darkgray}{rgb}{.4,.4,.4}
\definecolor{purple}{rgb}{0.65, 0.12, 0.82}
\lstdefinelanguage{JavaScript}{
  keywords={typeof, new, true, false, catch, function, return, null, catch, switch, var, if, in, while, do, else, case, break},
  keywordstyle=\color{blue}\bfseries,
  ndkeywords={class, export, boolean, throw, implements, import, this},
  ndkeywordstyle=\color{darkgray}\bfseries,
  identifierstyle=\color{black},
  sensitive=false,
  comment=[l]{//},
  morecomment=[s]{/*}{*/},
  commentstyle=\color{purple}\ttfamily,
  stringstyle=\color{red}\ttfamily,
  morestring=[b]',
  morestring=[b]"
}
\lstset{
   language=JavaScript,
   backgroundcolor=\color{lightgray},
   extendedchars=true,
   basicstyle=\footnotesize\ttfamily,
   showstringspaces=false,
   showspaces=false,
   numbers=left,
   numberstyle=\footnotesize,
   numbersep=9pt,
   tabsize=2,
   breaklines=true,
   showtabs=false,
   captionpos=b
}
\author{Андрей Лушников}
\title{Разработка и реализация алгоритмов деформирования трехмерной геометрии анатомии человека на основе WebGL}
\date{Апрель 2012 год}

\begin{document}
\maketitle

\tableofcontents
\newpage

\section{Введение}
Каждый программный продукт на протяжении всех стадий жизни сопровождает много
разнообразных проблем. Это и проблемы его поддержки, и проблемы переносимости, и
проблемы контроля версий и устаревания. Эти проблемы характерны как для рядовых
программ, которые знакомы каждому пользователю, так и для специализированных
программ, используемых врачами в их ежедневной практике для оценки,
визуализирования и прогнозирования результатов пластических операций. Один из
современных подходов к распространению программного обеспечения, носящий
название ``Software as a Service'' - призван решить целый класс таких проблем.

\\
Software-as-a-Service (далее ``SAAS'') - модель использования программного
обеспечения, при которой программный продукт выполнен в виде Web-приложения, и
разработчик самостоятельно управляет его развитием. Пользователи имеют доступ к
приложению через сеть интернет, при этом они избавлены от затрат, связанных с
установкой, обновлением и поддержкой работоспособности оборудования и
работающего на нем программного обеспечения.

\\
В работе было решено использовать этот подход для создания средства визуализации
и изменения формы 3D моделей.  Цель выполненной работы - создать программу и веб
сервис для визуализации и изменения формы трехмерной модели с использованием
вычислений на сервере. Использование клиент-серверной системы дает возможность
перенести трудоемкие алгоритмы обработки геометрии на серверную часть,
значительно снизив требования для клиентских рабочих станций.

\section{Постановка цели}
Программное обеспечение, разработанное специально для врачей и решающее
различные медицинские задачи, зачастую обладает большим количеством недостатков.
Оно не кроссплатформенно, а его установка зачастую трудоемка. Оно предъявляет
дополнительные требования к рабочим станциям врачей, что сужает область
применения.

\\
Основной задачей данной дипломной работы было разработать программный редактор,
позволяющий загружать и редактировать трехмерную модель, и при этом полностью
следовать модели SAAS, что избавило бы его от большого количества недостатков,
присущих программному обеспечению в общем и медицинскому программному
обеспечению в частности. Тем не менее, необходимо было дополнительно исследовать
возможность переиспользования программных модулей, написанных на С++ для
desktop-приложений редакторов геометрий, в новом редакторе.

\section{Архитектура}
\subsection{Описание архитектуры}
\subsubsection{Клиент}
Для выполнения поставленных целей было решено использовать клиент-серверную
архитектуру. В роли клиента выступает Rich Internet Application, полностью
написанное на HTML5.0 и Javascript, для отображения графики используется
перспективная технология WebGL.

\\
WebGL - современная перспективная технология, позволяющая использовать мощность
графического ускорителя для рендеренга трехмерной графики в браузере. Технология
разрабатывается промышленным консорциумом Khronos Group, который
специализируется на выработке открытых стандартов интерфейсов программирования в
области создания и воспроизведения динамической графики и звука. Активное
участие в разработке и внедрении стандарта так же принимают разработчики
браузеров Apple Safari, Google Chrome, Mozilla Firefox, Opera,  а также
специалисты компаний AMD и NVidia. На данный момент технология поддерживается в
последних версиях браузеров Safari, Mozilla, Opera и Chrome, а так же в браузере
Internet Explorer вместе с плагином IEWebGL. Среди мобильных устройств данная
технология уже поддерживается в браузере аппарата Nokia N900, а так же в
браузере Safari Mobile начиная с версии операционной системы iOS 4.2
\footnote{стоит отметить, что использование WebGL разрешено в браузере Mobile
Safari только в контексте рекламных объявлений iAd}

\subsubsection{Сервер}
Cервеная часть была написана на платформе node.js. Эта технология позволяет
исполнять JavaScript на стороне сервера, причем в качестве движка используется
высокопроизводительный V8, разработанный компанией Google.

\\
Преимущества от использования динамического слабо-типизированного
прототипно-ориентированного языка на стороне сервера заметны далеко не сразу.
Однако использование одного и того же языка на стороне клиента и сервера
позволяет переиспользовать код (например, для проверки форм) и снижает затраты
по передаче данных между клиентом и сервером, т.к. код описания модели тоже может
быть переиспользован. Серверные скрипты на JavaScript по своей природе являются
полностью асинхронными, что упрощает масштабирование серверов на Node.js.

\subsubsection{Переиспользование С++ кода}
Для поддержания возможности переиспользования модулей и алгоритмов, написанных
на С++ в процессе работы над desktop-приложениями, используется технология
Apache Thrift. Эта технология была разработана в Facebook по аналогии с Google
Protocol Buffers и служит для написания приложений на нескольких языках
программирования. Thrift работает следующим образом: по описанной в специальном
файле и на специальном языке модели можно создать файлы описаний этих моделей
для любого из 20 языков, а по описанному на том же thrift-языке rpc-методу можно
сгенерировать thrift-сервер для интересующего языка, обрабатывающий вызов этого
метода.

\subsection{Альтернативы}
\subsubsection{Flash на стороне пользователя}
Рассказать почему Flash на стороне клиента не очень хорош как для создания
пользовательского интерфейса, так и в свете применения многочисленных
3D-движков. В то же время не забыть про его достоинства: например, по нему очень
много материалов и на нем просто сделать богатую и интересную графику.
\subsubsection{unity3D вместо webGL}
В принципе ок вариант: мощный такой 3Д движок, десктопная и мобильная версии
которого получили большое распространение. Однако мобильная версия требует
установки плагина, что не очень понравилось (по тем же причинам, что и flash)
\subsubsection{Ruby on Rails вместо node.js}
Хорошая штука! Но отдельные усилия нужны для создания и поддержания протокола
общения клиента с сервером (разные языки - конвертация!)
\subsubsection{Protocol Buffers вместо Apache Thrift}
Тут рассказать почему выбрали второе а не первое (есть в письме к Саше)

\section{Детали реализации}
\subsection{Клиентские технологии}
\subsubsection{Движок для 3D графики}
Тут надо написать про Three.js, про то, что он является оберткой над специальном
языком для шейдеров и не только, и что у него есть два уровня: уровень WebGL,
или уровень графической карты, на котором можно делать любые вещи, которые
позволены вершинным и пиксельным шейдерами, а так же уровнем js-кода, который
оборачивает это безобразие в высокоуровневые конструкции языка javascript

\subsubsection{Инструменты для обработки графики}
Тут надо написать про шаблон Strategy при использовании
тулов для обработки графики
\subsubsection{Тесселляция}
Тут надо написать про тесселляцию из движка three.js, про то
что она не очень здорово работала и что обсуждение проблем было поднято в
сообществе, однако результатов оно не дало
\subsubsection{Проблема поворота}
Тут надо написать про проблему с поворотом объекта, про разные
координатные системы и про решение этой проблемы ввиде перемножения
матриц
\subsubsection{Шина событий}
Тут надо рассказать про EventBus, про обеспечение двунаправленных
коммуникаций между модулями приложения, а так же почему этот подход
можно считать достаточно удачным.
\subsubsection{Инструмент ``Деформация''}
Тут надо написать про инструмент деформация, рассказать про его формулы,
а так же не забыть его доделать так, чтобы он не изменял те вершины, что
находятся в другом направлении от направления деформации
\subsubsection{Использование технологии AJAX}
Тут надо написать про то, что все клиент-серверные взаимодействия сделаны по
технологии AJAX, зачем это сделано, и что, в частности, пришлось пойти на
уступки и добавить поддержку полных путей для загрузки моделей через url для
удобства. Частая ошибка, которая все ломала: относительные, а не абсолютные
адреса ссылок у действий.

\subsection{Серверные технологии}
\subsubsection{Серверная платформа node.js}
Тут надо сказать пару слов об этой платформе, почему она такая хорошая,
какие ей есть альтернативы (тут heavy wiki!)
\subsubsection{Фреймворк Express.js}
Тут надо обосновать необходимость использования серверного фреймворка,
аргументируя решением таких задач, как маршрутизация и рендеринг представлений
\subsubsection{HTML препроцессор Jade}
Тут нужно обосноват использование html-препроцессора jade, рассказать про его
достоинства и про то, почему его удобно использовать (видимо, вики)
\subsubsection{Конвертация загруженных *.obj-объектов}
Рассказать, как это делается, что происходит, какие врапперы для чего были
написаны, как и где хранятся объекты

\subsection{Серверные вычисления}
\subsubsection{Формулировка задачи}
Тут надо еще раз кратко рассказать (еще раз - потому как один раз это уже должно было
быть в ведении) про большой codebase на C++ и про необходимость как-то
интегрировать этот код
\subsubsection{Применение thrift-технологии}
Рассказать про то, что есть один файл model.thrift, что он описывает типы
данных, а так же один сервис. Можно даже скопировать частично этот файл сюда, по
крайней мере код описания сервиса так точно. Что по этому файлу генерятся стабы
для С++ и для js, а так же скелет для С++ сервера. Что этот скелет потом
с помощью скрипта-генератора наполняется содержанием на основе содержимого папки
algo, и с помощью готового makefile'а можно быстро собрать готовый
thrift-сервер.
\subsubsection{Проблема с десериализацией 8-байтного вещественного типа}
Тут надо рассказать, что у node-thrift'a была проблема с десериализацией Double,
о том что она была локализована, устранена, и соответствующий патч был послан в
сообщество на рассмотрение
\subsubsection{Организация С++ кода}
Рассказать про то, что С++ код можно писать довольно-таки обособленно,
рассказать про проблему с идентификацией алгоритмов и с решением, включающим в
себя мета-информацию в комментариях к алгоритму, которая парсится специальным
скриптом. Рассказать про специальные ruby-скрипты для пользователя и их опции

\subsection{Деплоинг приложения}
Проблема деплоинга приложения, использование сначала сбоственной машины, а потом
облачного сервиса heroku.com и его стека приложений CEDAR для временного
хостинга приложения


\section{Результат}
\subsection{Общий результат}
Сделано бла-бла-бла, можно сказать, проведен эксперимент по возможности
использования WebGL и что его можно считать удачным.
\subsection{Проведенные тесты}
\subsubsection{Тесты производительности Three.js}
Надо рассказать про некоторое количество конфузов, которые возникли при работе
с движком. Например, про то, что пересечение луча со сценой работает на уровне
JavaScript'a и потому не очень скоростное, хотя и использует отсечения по
описывающей сфере
\subsubsection{Тесты производительности node-thrift}
Рассказать про неожиданное падение в скорости при сериализации объектов в
node-thrift, о некоторых замерах времени, а так же сказать, что это очень плохо
и хотелось бы это пофиксить. Можно зафигачить какую-нибудь табличку со
сравнительными данными.
\subsection{Выводы}
\end{document}
