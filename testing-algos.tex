\section{Тестирование}

Для того, чтобы установить перспективы использования разработанного web-сервиса,
была проведена серия экспериментов. Эксперименты замеряли одну из следующих
величин:
\begin{enumerate}
    \item FPS (Frames per second) - количество итераций отображения
    трехмерной сцены в секунду. Считается, что значения в 30FPS и более являются
    достаточно комфортными для человеческого восприятия.
    \item Время отклика сервера при использовании серверных
    вычислений. Вычисляется как временной промежуток между обновленным
    результатом на клиенте по выполнению серверных вычислений и непосредственной
    инициацией процесса на клиенте.
\end{enumerate}

Каждый эксперимент имел целью исследование зависимостей между количеством
полигонов в модели и тем или иным изучаемым параметром. Исследования проводились
на четырех различных моделях

\begin{center}
\scalebox{0.8}{%
    \begin{tabular}{ | c | c | c | c | c | c | }
    \hline
     Имя модели & \textbf{couch1} & \textbf{Jane\_solid\_obj} & \textbf{ladybird} & \textbf{Woman-head} & \textbf{Mini-cooper} \\ \hline
     Количество полигонов & 3098 & 12334 & 23496 & 114344 & 254714  \\
    \hline
    \end{tabular}
}
\end{center}

Эксперименты проводились на тестовом стенде следующей конфигурации:
\begin{itemize}
    \item Браузер: Google Chrome v.19.0
    \item Процессор: Intel Core i5, 1.7ГГц, кэш-память третьего уровня в 3 Мб
    \item Память: 4 ГБ DDR3 1333МГц
    \item Графический процессор: Intel HD Graphics 3000 с памятью 384
\end{itemize}

\subsection{Тест: просмотр модели}

Эксперимент проводился с целью определения возможностей статического
рендеринга трехмерных моделей средствами технологии WebGL и фреймворка
``Three.js''. В рамках исследования был проведен ряд тестов, каждый из которых
состоит из следующей последовательности шагов:
\begin{enumerate}
    \item В web-приложение загружается трехмерная модель
    \item С загруженной моделью совершают ряд операций поворота
    и масштабирования
    \item Результатом теста является среднее значение FPS, полученное
    за время работы с моделью
\end{enumerate}

\subsubsection{результаты}
\begin{center}
\scalebox{0.7}{%
    \begin{tabular}{ | c | c | c | c | c | c | }
    \hline
    Количество полигонов & 3098 & 12334 & 23496 & 114344 & 254714 \\
    \hline
    FPS & 60 & 60 & 60 & 60 & 59 \\
    \hline
    \end{tabular}
}
\end{center}

\subsubsection{выводы}

Из результатов эксперимента видно, что технология WebGL вместе с фреймворком
``Three.js'' отлично справляется с рендерингом трехмерных моделей.

\subsection{Тест: локальное изменение модели}

Эксперимент проводился с целью определения возможностей динамического
изменения трехмерных моделей на стороне клиента. Каждый из тестов, проведенный в
рамках исследований, состоит из следующей последовательности шагов:

\begin{enumerate}
    \item В web-приложение загружается трехмерная модель
    \item С загруженной моделью с помощью инструмента для локального изменения
    геометрии совершается ряд преобразований
    \item Результатом теста является среднее значение FPS, полученное за время
    работы с моделью
\end{enumerate}

\subsubsection{результаты}
\begin{center}
\scalebox{0.8}{%
    \begin{tabular}{ | c | c | c | c | c | c | }
    \hline
    Количество полигонов & 3098 & 12334 & 23496 & 114344 & 254714 \\
    \hline
    FPS & 60 & 41 & 22 & 6 & 2 \\
    \hline
    \end{tabular}
}
\end{center}

\subsubsection{выводы}

Эксперимент показала явную регрессию производительности при увеличении
количества полигонов в модели. Это связано в первую очередь с тем, что
инструмент локального изменения геометрии добавляет в процесс рендеринга
дополнительную операцию пересечения луча со сценой (операция необходима для
того, чтобы расположить схематичную сферу трансформации на модели).

Операция пересечения луча с объектом, в свою очередь, написана на языке
JavaScript и работает за $O(N)$, где $N$ - количество полигонов в объекте.
Именно такую зависимость FPS от количества полигонов можно наблюдать в
результате эксперимента.

Эксперимент демонстрирует неприменимость текущей реализации для
высоко-полигональных моделей. Можно перечислить следующие пути для решения этой
проблемы, которые могут быть реализованы в дальнейшем
\begin{enumerate}
    \item Путем изменения схемы взаимодействия пользователя с объектом можно
    исключить необходимость в отрисовке схематичной сферы деформации
    \item Заметим, что за единицу времени сфера трансформации может сдвинуться
    на не очень большое расстояние. Такой подход дает возможность для попыток
    реализации множественных инкрементальных алгоритмов, которые делают
    первоначальное размещение сферы за время $O(N)$, однако после этого
    незначительные изменения в положении сферы могут быть пересчитаны за гораздо
    меньшее количество операций.
\end{enumerate}

