\section{Тестирование}

Для того, чтобы установить перспективы использования разработанного web-сервиса,
была проведена серия экспериментов. Эксперименты замеряли одну из следующих
величин:
\begin{enumerate}
    \item FPS (Frames per second) - количество итераций отображения
    трехмерной сцены в секунду. Считается, что значения в 30FPS и более являются
    достаточно комфортными для человеческого восприятия.
    \item Время отклика сервера при использовании серверных
    вычислений. Вычисляется как временной промежуток между обновленным
    результатом на клиенте по выполнению серверных вычислений и непосредственной
    инициацией процесса на клиенте.
\end{enumerate}

Каждый эксперимент имел целью исследование зависимостей между количеством
полигонов в модели и тем или иным изучаемым параметром. Исследования проводились
на четырех различных моделях

\begin{center}
\scalebox{0.8}{%
    \begin{tabular}{ | c | c | c | c | c | c | }
    \hline
     Имя модели & \textbf{couch1} & \textbf{Jane\_solid\_obj} & \textbf{ladybird} & \textbf{Woman-head} & \textbf{Mini-cooper} \\ \hline
     Количество полигонов & 3098 & 12334 & 23496 & 114344 & 254714  \\
    \hline
    \end{tabular}
}
\end{center}

Эксперименты проводились на тестовом стенде следующей конфигурации:
\begin{itemize}
    \item Браузер: Google Chrome v.19.0
    \item Процессор: Intel Core i5, 1.7ГГц, кэш-память третьего уровня в 3 Мб
    \item Память: 4 ГБ DDR3 1333МГц
    \item Графический процессор: Intel HD Graphics 3000 с памятью 384
\end{itemize}

\subsection{Тест: серверные вычисления}

Эксперимент проводился с целью определения эффективности технологии серверных
вычислений.

\subsubsection{результаты}
\begin{table}[ht]
\caption{Серверные вычисления, 3G}
\begin{center}
\scalebox{1.0}{%
    \begin{tabular}{ | c | c | c | c | c | c | }
    \hline
    Количество полигонов & 3098 & 12334 & 23496 & 114344 & 254714 \\
    \hline
    LAN задержка, мс & 757 & 1569 & 4694 & 9166 & 34583 \\
    \hline
    3G задержка, мс & 2429 & 8753 & 4694 & 9166 & 34583 \\
    \hline
    \end{tabular}
}
\end{center}
\label{table:nonlin}
\end{table}

\subsubsection{выводы}

При анализе результатов эксперимента оказалось, что большая часть потраченного
времени уходила на процессы сериализации и передачи данных в модуле
``Node-Thrift''. Данная проблема уже замечена и решается разработчиком модуля,
что дает надежду на скорое ее разрешение в ближайшем будущем.

