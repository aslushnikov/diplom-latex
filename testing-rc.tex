\section{Тестирование серверных вычислений}

Для того, чтобы оценить возможности использования серверных вычислений, были
сделаны эксперименты по их выполнению с разными моделями. В качестве алгоритма,
исполняемого на сервере, выступал однопроходный алгоритм масштабирования.

\begin{lstlisting}[language=c++]
#include "scale_x2.h"
using namespace std;
using namespace threejs;

void scale_x2(Geometry& _return, const Geometry& geom) {
    vector<Vertex> vv = geom.vertices;
    for(vector<Vertex>::iterator it = vv.begin(); it != vv.end(); ++it) {
        Vertex v = *it;
        v.x *= 2;
        v.y *= 2;
        v.z *= 2;

        _return.vertices.push_back(v);
    }
}
\end{lstlisting}

Результатом каждого эксперимента была задержка - количество времени, прошедшее
с начала отправления зарпоса в браузере, до получения и применения ответа.
Каждый эксперимент проводился при условии соединения клиентской машины с сетью
Интернет как по локальной сети с помощью WiFi, так и через протокол передачи
данных 3G, что обусловленно нацеленностью проекта на мобильные технологии.

Сервер, обрабатывающий запросы, географически располагался в Германии, и имеет
следующие параметры
\begin{itemize}
    \item Процессор: 3073МГц
    \item Память: 500Мб
    \item Канал интернет: обеспечивает скорость более 2 Мегабайт/с на загрузку, и
    столько же на отдачу
\end{itemize}

В результате проведения экспериментов были получены результаты,
сведенные в единую таблицу ~\ref{table:nonlin}

\begin{table}[ht]
\begin{center}
\scalebox{1.0}{%
    \begin{tabular}{ | c | c | c | c | c | c | }
    \hline
    Количество полигонов & 3098 & 12334 & 23496 & 114344 & 254714 \\
    \hline
    LAN задержка, мс & 397 & 826 & 2394 & 4103 & 17956 \\
    \hline
    3G задержка, мс & 3344 & 8402 & 34018 & - & - \\
    \hline
    Взаимодействие node-thrift & 43.5 & 148.5 & 672 & 1278.5 & 5448.5 \\
    \hline
    \end{tabular}
}
\end{center}
\caption{Задержка серверных вычислений}
\label{table:nonlin}
\end{table}

При анализе результатов эксперимента оказалось, что несмотря на достаточно
хорошие результаты работы по LAN, использование приложения по 3G не
представляется возможным.
