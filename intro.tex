\section{Постановка задачи}
Каждый программный продукт на протяжении всех стадий жизни сопровождает много
разнообразных проблем. Это и проблемы его поддержки, и проблемы переносимости, и
проблемы контроля версий и устаревания. Эти проблемы характерны как для рядовых
программ, которые знакомы каждому пользователю, так и для специализированных
программ, используемых врачами в их ежедневной практике для оценки,
визуализирования и прогнозирования результатов пластических операций. Один из
современных подходов к распространению программного обеспечения, носящий
название ``Software as a Service'' - призван решить целый класс таких проблем.

Software-as-a-Service (далее ``SAAS'') - модель использования программного
обеспечения, при которой программный продукт выполнен в виде Web-приложения, и
разработчик самостоятельно управляет его развитием. Пользователи имеют доступ к
приложению через сеть интернет, при этом они избавлены от затрат, связанных с
установкой, обновлением и поддержкой работоспособности оборудования и
работающего на нем программного обеспечения.

Программное обеспечение, разработанное специально для врачей и решающее
различные медицинские задачи, зачастую также обладает большим количеством
недостатков.  Оно не кросс платформенное, а его установка зачастую трудоемка.
Оно предъявляет дополнительные требования к рабочим станциям врачей, что сужает
область применения.

Основной задачей данной дипломной работы было разработать программный редактор,
позволяющий загружать и обозревать трехмерную модель, а так же алгоритм,
позволяющий пользователю изменять части геометрии загруженной модели.
Разработанное приложение должно полностью следовать модели SAAS, что избавило бы
его от большого количества недостатков, присущих программному обеспечению в
общем и медицинскому программному обеспечению в частности. Программный редактор
должен был быть как можно менее зависим от конфигурации клиентской платформы, что
упростило бы его использование в больницах. Отдельным достоинством приложения
была бы возможность использования на планшетных компьютерах.

В новом редакторе необходимо было дополнительно исследовать и поддержать
возможность переиспользования программных модулей, написанных на С++ для ранее
разработанных desktop-приложений.  Реализация такой возможности позволяет не
только переиспользовать унаследованный от предыдущих проектов код, но и дает
возможность описывать и выполнять сложные вычисления на компилируемых и
хорошо-оптимизируемых языках программирования, что выливается в прирост
производительности.
