\subsubsection{Переиспользование унаследованного кода}

Для поддержания возможности переиспользования модулей и алгоритмов, написанных
на языке программирования С++ для ранее разработанных desktop-приложений,
используется технология Apache Thrift.

Эта технология была разработана в Facebook по аналогии с Google Protocol Buffers
и служит для написания приложений на нескольких языках программирования. Thrift
представляет собой транскомпилятор, преобразующий файл описания данных и
сигнатуры методов, предназначенных для удаленного вызова, на любой из
поддерживаемых языков\footnote{Thrift версии 0.8.0 поддерживает 20 языков}.

С помощью этой технологии серверная часть приложения получила возможность
вызывать методы, написанные на языках программирования, отличных от JavaScript.
Однако так как вся информация о состоянии геометрии находится на клиенте, то
отдельно было реализовано дополнительное клиент-серверное взаимодействие,
передающее текущую информацию о геометрии с клиента на сервер. Таким образом,
была реализована следующая схема

\begin{enumerate}
    \item Данные о состоянии геометрии на клиенте кодируются в соответствии с
    установленным форматом клиент-серверного взаимодействия и по протоколу HTTP
    передаются на сервер
    \item Сервер получает состояние геометрии и передает ее в качестве аргумента
    при вызове удаленного метода, в нашем случае реализованного на языке C++.
    За вызов удаленного метода отвечает технология Thrift
    \item Результаты работы метода возвращаются на сервер. Полученный ответ
    кодируется в соответствии с форматом клиент-серверного взаимодействия и по
    протоколу HTTP отправляются обратно на клиент.
    \item Клиент, получив ответ, применяет новую геометрию к объекту.
\end{enumerate}

